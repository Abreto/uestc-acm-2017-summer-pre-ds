\documentclass[fontset=windowsnew,zihao=-4,scheme=chinese,punct=quanjiao,linespread=1,UTF8]{ctexart}
\usepackage[a4paper,centering,top=3cm,left=2.6cm]{geometry}
\pagestyle{plain}
\ctexset{section={
    name={},
    aftername={ - },
    number=\Alph{section},
    format=\Large\bfseries%\sffamily\raggedright
    }
}
\begin{document}

\title{2017 UESTC Training for Data Structures}
\author{傅宣登 2016030102010}

\maketitle

\section{An easy problem A}

区间 $[L,R]$ 内的数的极差就是此区间内数的最大值减去最小值,即

\[
Q_{[L,R]} = \max_{L\le i \le R}{a_i} - \min_{L\le i \le R}{a_i}
\]

用 \verb|Sparse-table| 或线段树维护区间最大最小值即可.

时间复杂度为 $O(N\log{N}+Q)$ 或 $O(N+Q\log{N})$.

\section{问题二}

\begin{equation}
  \ln((n-1)!) < \int_1^n \ln t\,\mathrm{d}t < \ln(n!)
\end{equation}

\subsection{概论}


\today \figurename

\section{问题三}


\section{问题四}

\section{问题五}
\[
  \sum_{p\rm\;prime}f(p) = \int_{t>1}f(t)\mathrm{d}\pi(t).
\]

\section{问题六}

\section{问题七}

\section{问题八}

\section{问题九}

\section{问题十}

\section{问题十一}

\section{问题十二}

\section{问题十三}

\section{问题十四}

\section{问题十五}

\section{问题十六}

%\appendix

\end{document}
